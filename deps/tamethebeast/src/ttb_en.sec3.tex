%% ttb_en.sec3.tex
%% Copyright 2003-2005  Nicolas Markey
%
% This work may be distributed and/or modified under the
% conditions of the LaTeX Project Public License, either version 1.3
% of this license or (at your option) any later version.
% The latest version of this license is in
%   http://www.latex-project.org/lppl.txt 
% and version 1.3 or later is part of all distributions of LaTeX
% version 2003/12/01 or later.
% 
% This work has the LPPL maintenance status "maintained".
% The Current Maintainer of this work is Nicolas Markey  
% 
% This work consists of the files
%       ttb_en.tex
%       ttb_en.sec1.tex  ttb_en.sec2.tex  ttb_en.sec3.tex
%       ttb_en.sec4.tex  ttb_en.sec5.tex  ttb_style.sty
%       local.bib        idxstyle.ist     Makefile
% and the derived ttb_en.dvi, ttb_en.ps and ttb_en.pdf


\mypart{The \protect\niext{bib} file}\label{part3}
\parttoc
\mtcskip

The \ext{bib} file is the database. Its contents heavily
depends on the style being applied to it, even though 
bibliography styles are generally ``compatible'' with each w.r.t.\ the
database. I will only describe here
the case of standard styles\footnote{Thus using entry types
and fields as described at pages~\pageref{champ} 
to~\pageref{fin-champ}.}. 
But remember that
\bt can do many other things, we'll see some examples in
section~\ref{autres}. 

\mysection{Structure of the \protect\niext{bib} file}

We start with  an example:

\begin{verbatimtab}
@book{companion,
  author 	= "Goossens, Michel and Mittelbach, Franck and Samarin, Alexander",
  title 	= "The {{\LaTeX}} {C}ompanion",
  publisher 	= "Addison-Wesley",
  year 		= 1993,
}
\end{verbatimtab}
We just consider the structure of this entry for the moment, not its
contents. The general form is the following\footnote{The 
outermost braces can be replaced by parentheses: 
 \texttt{\at{\itshape entry\_type}({\itshape internal\_key}, ...)}.}:
\bigskip

\noindent\begin{minipage}{\textwidth}
\obeylines
\ttfamily \at{\itshape entry\_type}\{internal\_key, 
\ \ {\bfseries field\_1}\ \ \ \ \ \ \ = "value\_1",
\ \ {\bfseries field\_2}\ \ \ \ \ \ \ = "value\_2",
\ \ ...
\ \ {\bfseries field\_n}\ \ \ \ \ \ \ = "value\_n"
\}
\end{minipage}

\bigskip

\noindent Some basic remarks:
\begin{itemize}
\item New entries always start with \texttt\at. Anything outside
  the ``argument'' of a ``command'' starting with an \texttt\at{} is
  considered as a comment. This gives an easy way to comment a given
  entry: just remove the initial \texttt\at.
  As usual when a language allows comments, don't hesitate to use
  them so that you have a clean, ordered, and easy-to-maintain
  database. Conversely, anything starting with an \texttt\at{} is
  considered as being a new entry\footnote{There is a special entry
  type named \ent{comment}. The main use of such an
  entry type is to comment a large part of the bibliography easily,
since anything outside an entry is already a comment, and commenting 
out one entry may be achieved by just removing its 
initial~\texttt{\at}.}.

\item \bt does not distinguish between normal and capital
  letters in entry and field names. \bt will complain if two entries
  have the same internal key, even if they aren't capitalized in the
  same way. For instance, you cannot have two entries named
  \texttt{Example} and  \texttt{example}. 

  In the same way, if you cite both \texttt{example} and
  \texttt{Example}, \bt will complain. Indeed, it~would have to
  include the same entry twice, which probably is not what you want;


\item Spaces and line breaks are not important, except for 
  readability. On the contrary, commas are compulsory between any two
  fields;


\item Values (\emph{i.e.} right hand sides of each assignment) can be either
  between curly braces or between double quotes. The main difference
  is that you can write double quotes in the first case, and not in
  the second case. For citing 
  \emph{Comments on ``Filenames and Fonts''} by Franck
\textsc{Mittelbach}, you can use one of the following solutions:
\begin{verbatimtab}
  title 	= "Comments on {"}Filenames and Fonts{"}",
  title 	= {Comments on "Filenames and Fonts"},
\end{verbatimtab}

Curly braces have to match, since they will appear in the output to be
compiled by \LaTeX. 
A problem occurs if you need to write a (left-, say) brace in an entry. You
could of course write \verb+\{+, but the entry will have to also include the
corresponding right brace. To include a left brace without its
corresponding right brace, you'll have use a \LaTeX~function having no
brace in its name. \cmd{leftbrace} is the right choice here. Another solution
is to add an extra \cmd{bgroup} in the entry, so that both \LaTeX and \bt will
find the correct number of ``braces''.

\item For numerical values, curly braces and double quotes can be
  omitted.
\end{itemize}


As I already mentioned, you can define fields even if they aren't
used by the style being applied. For instance, the following can be
used for the \LaTeX Companion:
\begin{verbatimtab}
@book{companion,
  author 	= "Goossens, Michel and Mittelbach, Franck and Samarin, Alexander",
  title 	= "The {{\LaTeX}} {C}ompanion",
  booktitle 	= "The {{\LaTeX}} {C}ompanion",
  publisher 	= "Addison-Wesley",
  year 		= 1993,
  month		= "December",
  ISBN 		= "0-201-54199-8",
  library 	= "Yes",
}
\end{verbatimtab}
It gives complementary information about the book\footnote{Filling
both \nichp{title} and \nichp{booktitle} for a book may be of real
interest, as we will see in the section dedicated to cross references
(see section~\ref{crossref} for some more details).}, for instance the
fact that it is available in your local library.  You really should
not hesitate to use auxiliary, personal fields, giving them explicit
names in order to be sure that no bibliography style will use them
incidentally\footnote{You'll ask ``Is it really useful?''. First of
all, it doesn't take long to add those informations each time you add a
new entry; it~takes much longer to add a field to several
entries. Moreover, we'll see later how to design bibliography styles,
and you'll be able to write styles that take those new fields into
account.}.


\mysection{The  \protect\nient{string} and \protect\nient{preamble}
  entries}\label{strpre}

These are not really entry types: \ent{string}
entries can be used in order to define abbreviations. For instance,
we've cited two books published by Addison-Wesley. It might be useful
to define a shortcut for this publisher. Thus we write:
\begin{verbatimtab}
@string{AW	= "Addison-Wesley"}

@book{companion,
  author 	= "Goossens, Michel and Mittelbach, Franck and Samarin, Alexander",
  title 	= "The {{\LaTeX}} {C}ompanion",
  booktitle 	= "The {{\LaTeX}} {C}ompanion",
  publisher 	= AW,
  year 		= 1993,
  month		= "December",
  ISBN 		= "0-201-54199-8",
  library 	= "Yes",
}
\end{verbatimtab}

This does not only spare some time, but most importantly, it ensures that
you won't misspell it, and helps you maintain a homogeneous database.
I find this really interesting for author names, so that you're sure
to always write them correctly (or always incorrectly, but it would 
be easy to detect and correct a global mistake anyway).


As regards \ent{preamble}, it may be used for inserting commands or
text in the file created by \bt. Anything declared in a \ent{preamble}
command will be concatenated and put in a variable named
\fn{preamble}, for being used in the bibliography style and,
generally, inserted at the beginning of the \ext{bbl} file, just
before the \env{thebibliography} environment. This is useful for
defining new commands used in the bibliography. Here is a small
example:
\begin{verbatimtab}
@preamble{ "\makeatletter" }
@preamble{ "\@ifundefined{url}{\def\url#1{\texttt{#1}}}{}" }
@preamble{ "\makeatother" }
\end{verbatimtab}
This way, you may safely use the \nicmd{url} command in your
entries. If it is not defined at the beginning of the bibliography, the
default command defined in the \ent{preamble} will be used.  

Please note that you should never define style settings
in the \ent{preamble} of a bibliography database, 
since it would be applied to
any bibliography built from this database.






\mysection{The \nichp{title} field}\label{title}

Let's see how to fill the ``title'' field. We start by studying how I entered
the title field for the  
\emph{\LaTeX Companion} : 
\begin{verbatimtab}
  title 	= "The {{\LaTeX}} {C}ompanion"
\end{verbatimtab}

We'll need several definitions before going further.
The \emph{brace depth} of an item is the number of braces surrounding
it. This is not a very formal definition, but for instance, in
the title above, \verb+\LaTeX+ has brace depth $2$, the \verb+C+ has
brace depth $1$, and everything else has depth $0$\footnote{Of course,
surrounding the field with braces instead of quotes does not modify
the brace depth.}.
A~\emph{special character} is a part of a field starting with a left 
brace
being at brace depth $0$ immediately 
followed with a backslash, and ending 
with the corresponding right brace. 
For instance, in the above example, there is
no special character, since \verb+\LaTeX+ is at depth~$2$. It should
be noticed that anything in a special character is considered as being at
brace depth~$0$, even if it is placed between another pair of braces.

That's it for the definitions.
Generally speaking, several modifications can be applied to the
title by the bibliography style:
\begin{itemize}
\item first of all, the title might be used for sorting. When sorting,
  \bt computes a string, named \fn{sort.key}, for each
  entry. The \fn{sort.key} string is an (often long) 
  string defining the order in which
  entries will be sorted. To avoid any ambiguity, \fn{sort.key} should
  only contain alphanumeric characters. Classical non-alphanumeric 
  characters\footnote{Except hyphens and tildes 
  that are replaced
  by spaces. Spaces are preserved. The precise behavior of 
  \fn{purify} is explain on page~\pageref{purify}.}, 
  except special characters, will be removed by 
  a \bt function named \fn{purify}. For special characters, \fn{purify}
  removes spaces and 
  \LaTeX commands (strings beginning with a backslash), even those
  placed between brace pairs. Everything else is left unmodified. For
  instance, 
  \verb+t\^ete+, \verb+t{\^e}te+ and \verb+t{\^{e}}te+
are transformed into \verb+tete+, 
while \texttt{t\^ete} gives \texttt{t\^ete};
\verb+Bib{\TeX}+ gives \verb+Bib+ and
\verb+Bib\TeX+ becomes \verb+BibTeX+.
There are thirteen \LaTeX commands that won't follow the above rules:
\verb+\OE+, \verb+\ae+, \verb+\AE+, \verb+\aa+, \verb+\AA+, 
\verb+\o+, \verb+\O+, \verb+\l+, \verb+\L+, \verb+\ss+.
Those commands correspond to \i, \j, \oe,
\OE, \ae, \AE, \aa, \AA, \o, \O,
\l, \L, \ss, and \fn{purify} transforms them (if they are in a special
character, 
in \verb+i+, \verb+j+,
  \verb+oe+, 
\verb+OE+, \verb+ae+, \verb+AE+, \verb+aa+, \verb+AA+, \verb+o+,
  \verb+O+, 
\verb+l+, \verb+L+, \verb+ss+, respectively. 


\item the second transformation applied to a title is to be turned to
  lower case (except the first character). 
  The function named
  \fn{change.case} does this job. But it only applies to letters that
  are a brace depth~$0$, except within a special character. In a
  special character, brace depth is always $0$, and letters are
  switched to lower case, except \LaTeX commands, that
  are left unmodified.
\end{itemize}
Both transformations might be applied to the \chp{title} field by 
standard styles, and you must ensure that your title will be treated correctly
in all cases. 

Let's try to apply it now, 
for instance to \emph{the \LaTeX Companion}. 
Several solutions might be tried:
\begin{itemize}
\item \verb+ title = "The \LaTeX Companion"+: This won't work, since
  turning it to lower case will produce \verb+The \latex companion+,
  and 
  \LaTeX won't accept this...
\item \verb+ title = "The {\LaTeX} {C}ompanion"+ : This ensures that
  switching to lower case will be correct. However, applying
  \fn{purify} gives  \verb+The  Companion+. Thus sorting could be
  wrong; 
\item \verb+ title = "The {\csname LaTeX\endcsname} {C}ompanion"+:
  This won't work since \texttt{LaTeX} will be turned to \texttt{latex};
\item \verb+ title = "The { \LaTeX} {C}ompanion"+ : In this case,
  \verb+{ \LaTeX}+ is \emph{not} a special character, but a set of
  letters at depth~$1$. It won't be modified by
  \fn{change.case}. However, \fn{purify} will leave both spaces, and
  produce \verb+The  LaTeX Companion+, which could result in wrong
  sorting;
\item \verb+ title = "The{ \LaTeX} {C}ompanion"+: This solution also
  works, but is not as elegant as the next one;
\item \verb+ title = "The {{\LaTeX}} {C}ompanion"+ : This is the
  solution I used. It solves the problems mentioned above.
\end{itemize}

For encoding an accent in a title, say \emph{\'E} (in upper case) as
in the French word \emph{\'Ecole}, we'll write 
\verb+{\'{E}}cole+, \verb+{\'E}cole+ or \verb+{{\'E}}cole+,
depending on whether we want it to be turned to lower case (the first
two solutions) or not (the last one). \fn{purify} will give the same
result in the three cases.
However, it should be noticed that the third one is not a special
character. If you ask \bt to extract the first character of each
string using \fn{text.prefix}, you'll get \verb+{\'{E}}+ in the first case,
\verb+{\'E}+ in the second case  and 
\verb+{{\}}+ in the third case. 


That's all for subtleties of titles. Let's have a look at author
names, which is even more tricky.


\mysection{The \nichp{author} field}\label{author}


We still begin with the entry for the \emph{\LaTeX Companion}: 
\begin{verbatim}
  author 	= "Goossens, Michel and Mittelbach, Franck and Samarin, Alexander"
\end{verbatim}
The first point to notice is that two authors are separated with the
keyword  \texttt{and}\index{bibtex}{and}. The format of the names is
the second important point: The last name first, then the first name, 
with a separating comma. In fact, \bt understands other formats.

Before going further, we remark an important point: \bt will have to
\emph{guess} which part is the first name and which part is the last
name. It also has to distinguish a possible ``von'' part (as in John
von Neumann) and a possible ``Jr'' part.

The following explanation is somewhat technical. The first name will
be called \verb+First+, the last name is denoted by \verb+Last+, the
``von'' with \verb+von+ and the ``Jr'' part, \verb+Jr+.


So, \bt must be able to distinguish between the different parts of the
\chp{author} field. To that aim, \bt recognizes three
possible formats:
\begin{itemize}
\item \verb+First von Last+; 
\item \verb+von Last, First+;
\item \verb+von Last, Jr, First+.
\end{itemize}

The format to be considered is obtained by counting the number of
commas in the name.
 Here are the characteristics of these formats:

\begin{itemize}
\item \verb+First von Last+: Suppose you entered
\verb+Jean de La Fontaine+. There is no comma, hence the format is 
\verb+First von Last+. The \verb+Last+ name cannot be empty, unless
the whole field is. It should then contain at least \verb+Fontaine+. 
\bt then looks at the first character\footnote{In the sequel, the
  first character means ``the first non-brace character that at brace depth
  $0$, if 
  any, characters of a special character being
  at depth $0$, even if there are $15$ braces around.'' If there is 
  no 
  character at depth $0$, then the item will go with its neighbour,
  first and foremost with the \texttt{First}, then with the \texttt{Last}. It 
  will be in the \texttt{von} if, and only if, it is surrounded with
  two \texttt{von} items. Moreover, two words in the same group
  (in \LaTeX sense) will go to the same place. Last, for a
  \LaTeX command outside a special character, the backslash is removed
  and \bt considers the remaining word. If you did not understand,
  please take a while for reading this note anew, since it will be
  used 
  in the sequel.} of each remaining word. If some 
of them are lower cases alphabetic letters, 
anything between the first and the last ones
(beginning with lower cases) is considered as being in the
\verb+von+. Anything before the 
\verb+von+ is in the \verb+First+, anything after is in the
\verb+Last+. If no first letter is in lower case, then everything
(except the part already put in the \verb+Last+) is put in the
\verb+First+. 

\verb+Jean de La Fontaine+ will then give \verb+La Fontaine+ as the
\verb+Last+, \verb+Jean+ for the \verb+First+ and \verb+de+ for the
\verb+von+. 

Here is what it gives for several other combinations. This is for you
to check if you understood:
\begin{center}
\begin{longtable}{|>{\vrule height 4mm depth 2mm width 0pt}p{.3\textwidth}|p{.15\textwidth}|p{.15\textwidth}|p{.18\textwidth}|}
\hline
Name & \verb+First+ & \verb+von+ & \verb+Last+ \endfirsthead
\hline
Name & \verb+First+ & \verb+von+ & \verb+Last+ \endhead
\hline
\verb+jean de la fontaine+ & \verb++ & \verb+jean de la+ & \verb+fontaine+ \\
\hline
\verb+Jean de la fontaine+ & \verb+Jean+ & \verb+de la+ & \verb+fontaine+ \\
\hline
\verb+Jean {de} la fontaine+ & \verb+Jean de+ & \verb+la+ & \verb+fontaine+ \\
\hline
\verb+jean {de} {la} fontaine+ & \verb++ & \verb+jean+ & \verb+de la fontaine+ \\
\hline
\verb+Jean {de} {la} fontaine+ & \verb+Jean de la+ & \verb++ & \verb+fontaine+ \\
\hline
\verb+Jean De La Fontaine+ & \verb+Jean De La+ & \verb++ & \verb+Fontaine+ \\
\hline
\verb+jean De la Fontaine+ & \verb++ & \verb+jean De la+ & \verb+Fontaine+ \\
\hline
\verb+Jean de La Fontaine+ & \verb+Jean+ & \verb+de+ & \verb+La Fontaine+ \\
\hline
\end{longtable}
\end{center}
The last line is the only one (in this table) that is correct.
Of course, some of you will have a counter example where the
\verb+von+ has to begin with an upper case letter. We'll see this at section~\ref{trucsbib}.

\item \verb+von Last, First+: The idea is similar, but identifying
  the \verb+First+ is easier: It's everything after the comma. Before
  the comma, the last word is put in the \verb+Last+ (even if it
  starts with a lower case). If any other word begins with a lower
  case, anything from the first word to the last one starting with a
  lower case is in the \verb+von+, and what remains is in the
  \verb+Last+. 
Once again, an example should make everything clear:

\begin{center}
\begin{longtable}{|>{\vrule height 4mm depth 2mm width 0pt}p{.3\textwidth}|p{.15\textwidth}|p{.15\textwidth}|p{.18\textwidth}|}
\hline
Name & \verb+First+ & \verb+von+ & \verb+Last+ \endfirsthead
\hline
Name & \verb+First+ & \verb+von+ & \verb+Last+ \endhead
\hline
\verb+jean de la fontaine,+\footnotemark & \verb++ & \verb+jean de la+ & \verb+fontaine+ \\
\noalign{\footnotetext{This case raises an error message from \bt, complaining
  that a name ends with a comma. 
It is a common error to separate names with commas instead of ``\texttt{and}''.}}%
\hline
\verb+de la fontaine, Jean + & \verb+Jean+ & \verb+de la+ & \verb+fontaine+ \\
\hline
\verb+De La Fontaine, Jean+ & \verb+Jean+ & \verb++ & \verb+De La Fontaine+ \\
\hline
\verb+De la Fontaine, Jean+ & \verb+Jean+ & \verb+De la+ & \verb+Fontaine+ \\
\hline
\verb+de La Fontaine, Jean+ & \verb+Jean+ & \verb+de+ & \verb+La Fontaine+ \\
\hline
\end{longtable}
\end{center}
%\medskip

\item \verb+von Last, Jr, First+: Well... It's still the same, except
  that anything between the commas is put in the \verb+Jr+.
\end{itemize}

Names are separated by spaces above, but it may occur that two
first names are separated by a hyphen, as in ``Jean-Fran\c cois'' 
for instance. \bt splits that string, and if both parts are in the
\verb+First+, the abbreviated surnames is ``J.-F.'' as (generally)
wanted. A tilde is also seen as a string separator. This all boils 
down to the fact that, if you enter \verb+Jean-baptiste Poquelin+,
the string \verb+baptiste+ will be in the \verb+von+ part of the 
name, since it erroneously begins with a lower case letter.
\medskip

I think it's a good place for coming back to abbreviations: You probably agree 
that names are something not that easy to enter, and are
error-prone. I personally advise defining an abbreviation for each
author. You'll then concatenate them with \verb+``and''+ using
\verb+#+\index{bibtex}{\#@\texttt{\#}(concatenation)}\index{bibtex}{concatenation@concatenation (\texttt{\#})} (actually, even \verb+``and''+ will be defined as an abbreviation). 

For instance, I always include a \ext{bib} file containing the
following lines:
\begin{verbatimtab}
@string{and = "and"}
...
@string{goossens    	= "Goossens, Michel"}
@string{mittelbach  	= "Mittelbach, Franck"}
@string{samarin		= "Samarin, Alexander"}
\end{verbatimtab}
another one containing:
\begin{verbatimtab}
@string{AW		= "Addison-Wesley"}
\end{verbatimtab}
and my main bibliographic file contains\footnote{This is not quite
  true, since \texttt{December} still depends on the language, and I
  prefer using \texttt{12} and letting the style file translate that
  into the corresponding month in the correct language.}:
\begin{verbatimtab}
@book{companion,
  author	= goossens # and # mittelbach # and # samarin,
  title		= "The {{\LaTeX}} {C}ompanion",
  booktitle	= "The {{\LaTeX}} {C}ompanion",
  year 		= 1993,
  publisher	= AW,
  month		= "December",
  ISBN 		= "0-201-54199-8",
  library 	= "Yes",
}
\end{verbatimtab}
This makes adding an entry much easier when you're used to such an
encoding. Moreover, you can easily recover self-contained entries by using
\texttt{bibexport.sh}~tool (see section~\ref{bibexport}).



\mysection{Cross-references (\protect\nichp{crossref})}\label{crossref} 
As mentioned earlier (can't remember where... Oh, yes, on
page~\pageref{cetaitla}), \bt allows for cross-referencing. This is
very useful, for instance, when citing a part of a book, or an article
in conference proceedings. For instance, for citing chapter~$13$ of
the \LaTeX~Companion:
\begin{verbatimtab}
@incollection{companion-bib,
  crossref 	= "companion",
  title		= "Bibliography Generation",
  chapter	= 13,
  pages		= "371-420",
}
\end{verbatimtab}
This shows why having defined the \chp{booktitle} field
of the \verb+companion+~entry is useful: It is not used in the \ent{book} entry,
but 
it is inherited in the above \ent{incollection} entry. We could of
course have added it by hand, but we should have added it in each
chapter we cite.

The other possible interesting feature is that, when cross-referencing
several times an entry that is not cited by itself, \bt can
``factor'' it, \emph{i.e.} add it in the list of references and
explicitly \cmd{cite} it in each entry. On the other hand, if the cross
reference appears only once, it inherits from the fields of the
reference, which is not included in the bibliography. 
Here is an example of both behaviours:

\medskip
\begin{myex}
\begin{thebibliography}{1}

\bibitem{bib}
Michel Goossens, Franck Mittelbach, and Alexander Samarin.
\newblock Bibliography generation.
\newblock In {\em The {{\LaTeX}} {C}ompanion\/} \cite{companion4}, chapter~13,
  pages 371--420.

\bibitem{math}
Michel Goossens, Franck Mittelbach, and Alexander Samarin.
\newblock Higher mathematics.
\newblock In {\em The {{\LaTeX}} {C}ompanion\/} \cite{companion4}, chapter~8,
  pages 215--258.

\bibitem{ind}
Michel Goossens, Franck Mittelbach, and Alexander Samarin.
\newblock Index generation.
\newblock In {\em The {{\LaTeX}} {C}ompanion\/} \cite{companion4}, chapter~12,
  pages 345--370.

\bibitem{companion4}
Michel Goossens, Franck Mittelbach, and Alexander Samarin.
\newblock {\em The {{\LaTeX}} {C}ompanion}.
\newblock Addison-Wesley, December 1993.

\end{thebibliography}
\end{myex}

\medskip \noindent or\medskip

\begin{myex}
\begin{thebibliography}{1}

\bibitem{bib2}
Michel Goossens, Franck Mittelbach, and Alexander Samarin.
\newblock Bibliography generation.
\newblock In {\em The {{\LaTeX}} {C}ompanion}, chapter~13, pages 371--420.
  Addison-Wesley, December 1993.

\bibitem{math2}
Michel Goossens, Franck Mittelbach, and Alexander Samarin.
\newblock Higher mathematics.
\newblock In {\em The {{\LaTeX}} {C}ompanion}, chapter~8, pages 215--258.
  Addison-Wesley, December 1993.

\bibitem{ind2}
Michel Goossens, Franck Mittelbach, and Alexander Samarin.
\newblock Index generation.
\newblock In {\em The {{\LaTeX}} {C}ompanion}, chapter~12, pages 345--370.
  Addison-Wesley, December 1993.

\end{thebibliography}
\end{myex}

In order to have one presentation or the other, we can tell \bt the
number of cross-references needed for an entry to be explicitly
\cmd{cite}{}'d. We use the \verb+-min-crossrefs+ command line argument
of \bt for that. 
For the first example above, I used
\verb+bibtex biblio+, 
while the second example was obtained with
\verb+bibtex -min-crossrefs=5 biblio+.


One other important remark is that cross-referenced entries must be
defined \emph{after} entries containing the corresponding
\chp{crossref} field. And you can't embed
cross-references, that is, you cannot \chp{crossref} an entry that
already contains a \chp{crossref}.


Last, the \chp{crossref} field has a particular behaviour: it always
exists, whatever the bibliography style. If a \chp{crossref}{}'ed entry
is included (either by a \cmd{cite} command or by \bt if there are
sufficiently many \chp{crossref}{}s), then the entries cross-referencing
it have their original \chp{crossref} field. Otherwise, that field is
empty. 



\mysection{Quick tricks}\label{trucsbib}

\mysubsection{How to get \emph{Christopher} abbreviated as \emph{Ch.}?}

First names abbreviation is done when extracting names from the \chp{author}
field. If the function is asked to abbreviate the first name (or any other part
of it), it will return the first character of each ``word'' in that part. 
Thus \emph{Christopher} gets abbreviated into \emph{C.} Special characters
  provide a solution against this problem: If you enter
  \verb+{\relax Ch}ristopher+, the abbreviated version would be
  \verb+{\relax Ch}.+, which gives \emph{Ch.}, while the long version is
  \verb+{\relax Ch}ristopher+, \emph{i.e.} \emph{{Ch}ristopher}. 

\mysubsection{How to get caps in \texttt{von}?}

Note: This part is somewhat technical. You should probably read and understand
how \bt extracts names, which is explained at pages~\pageref{noms-start}
and~\pageref{noms-end}. 

It may occur that the \verb+von+ part of a name begins with a capital letter.
For some reason, the standard example is \emph{Maria De La Cruz}. 

The basic solution is to write
\verb+"{\uppercase{d}e La} Cruz, Maria"+. When analyzing this name, \bt will
place \verb+Cruz+ in the
\verb+Last+ part, then \verb+Maria+ as the \verb+First+ name. Then
\verb+{\uppercase{d}e La}+ is a special character,  
whose first letter is \verb+d+, and is thus placed in the \verb+von+ part.

In that case, however, if you use an ``alphanumeric'' style such as
\bst{alpha}, \bt will use the first character of the
\verb+von+ part in the label\footnote{It could be argued \texttt{von} parts
should not be used when computing the label, but classical style file do use
it.}. But the first character here is
\verb+{\uppercase{d}e La}+, and the label would be
\verb+{\uppercase{d}e La}C+, and you'll get \texttt{[{\uppercase{d}e La}C]}.
You'd probably prefer \texttt{[DLC]} or \texttt{[Cru]}. The second solution
would then be 
\begin{verbatimtab}
  author = "{\uppercase{d}}e {\uppercase{l}}a Cruz, Maria"
\end{verbatimtab}
The label would be \verb+{\uppercase{d}}{\uppercase{l}}C+, which is correct.  
Another (easier) proposal would be
\begin{verbatimtab}
  author = "{D}e {L}a Cruz, Maria"
\end{verbatimtab}
This also solves the problem, because \bt only considers characters at level~0 when
determining which part a word belongs to, but takes all letters into account
when extracting the first letter.


\mysubsection{How to get lowercase letters in the \texttt{Last}?}

This is precisely the reverse problem, but the solution will be different. 
Assume you cite a paper by the famous Spanish scientist
\emph{Juan de la Cierva y Codorn\'\i u}. The basic ideas are
\begin{verbatimtab}
  author = "de la Cierva {\lowercase{Y}} Codorn{\'\i}u, Juan"
\end{verbatimtab}
or 
\begin{verbatimtab}
  author = "de la Cierva {y} Codorn{\'\i}u, Juan"
\end{verbatimtab}
However, these solutions yields labels such as \verb+CYC+ or \verb+CyC+, where
we would prefer \verb+CC+.

Several solutions are possible:
\begin{verbatimtab}
  author = "de la Cierva{ }y Cordon{\'\i}u, Juan"
\end{verbatimtab}
or
\begin{verbatimtab}
  author = "de la {Cierva y} Cordon{\'\i}u, Juan"
\end{verbatimtab}
Both solutions work: In the first case, \bt won't see the space, and
considers that the \verb+y+ belongs to the previous word. In the second case, 
\verb+Cierva y+ is at brace-level~$1$, and thus goes into the \verb+Last+ part,
which has priority over the \verb+von+ part.

\mysubsection{How to remove space between \texttt{von} and \texttt{Last}?}

Here the example will be \emph{Jean d'Ormesson}. 
The best way to encode this appears to be
\begin{verbatimtab}
  author = "d'\relax Ormesson, Jean"
\end{verbatimtab}
Indeed, the \nicmd{relax} commands will gobble spaces until the next
non-space character. 

\mysubsection{How to get \emph{et al.} in the author list?}

Some special bibliography styles will automatically replace long lists of 
authors with the name of the first author, followed by \emph{et al.} 
However, standard \bt styles won't. You can however get the same result by 
using the special name \verb+others+. For instance, if you enter
\begin{verbatimtab}
  author = "Dupont, Jean and others"
\end{verbatimtab}
you get "Jean Dupont \emph{et al.}" in the resulting bibliography.

\mysubsection{The \protect\nichp{key} field}

It may happen that no author is given for a document. In that case, 
bibliography styles such as \bst{alpha}, when computing the ``label'' of such
an entry, will use the \chp{key} field (the first three letters for
\bst{alpha}, but the entire field for \bst{apalike}, for instance). 
When no \texttt{key} is given, the first three letters of the internal
citation key are used.

\bigskip

Is it ok with everyone? Yes ?!
Right, we can go to the next, most exciting part of
this doc: How to create or modify a bibliography style...


